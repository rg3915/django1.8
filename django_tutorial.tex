\documentclass[aspectratio=169]{beamer}
\usepackage{graphics,amssymb,amsfonts,amsmath}
\usepackage{pgf,tikz}
\usetikzlibrary{shapes,shapes.geometric,positioning}
\DeclareGraphicsExtensions{.jpg,.pdf,.mps,.png}
\usepackage[latin1]{inputenc}
\usepackage[brazil]{babel}
\usepackage[normalem]{ulem}
\usepackage{standalone}
\usepackage{pgfpages,enumerate,hyperref}
\usepackage{palatino}   %Fonte sem serifa.
\usepackage{ragged2e}   %Par\'agrafo justificado.
\usepackage{minted}
% \usetheme{CambridgeUS}
% \usetheme{AnnArbor}
% \usecolortheme{lily}
\usecolortheme{orchid}
\usefonttheme[onlymath]{serif}

%colocando número de p\'aginas no slide.
\setbeamertemplate{footline}[frame number]

% desativando os botoes de navegacao.
\beamertemplatenavigationsymbolsempty

%Tela cheia
\hypersetup{pdfpagemode=FullScreen}

% Layout da p\'agina
\hypersetup{pdfpagelayout=SinglePage,urlcolor=blue,colorlinks=true}

% Ambiente bash
\newminted{bash}{bgcolor=gray}

\title{Tutorial Django 1.8}
\author{R\'egis da Silva {\texorpdfstring{\color{blue}}{ }@rg3915}\\ e\\ Jayme Neto \texorpdfstring{\color{blue}}{ }@kalkehcoisa}
\institute{Grupy-SP}
\date{\today}

\begin{document}
\justifying %Par\'agrafo justificado.

%Neste caso insere somente no primeiro slide.
{%
%\usebackgroundtemplate{\centering \vspace*{5cm} \includegraphics[width=\paperwidth]{figuras/djangoPython}}

\usebackgroundtemplate{%
\vbox to \paperheight{\vfil\hbox to \paperwidth{\hfil\includegraphics[width=\paperwidth]{img/logo_mascote.jpg}\hfil}\vfil}
}

\begin{frame}

\end{frame}
}

\begin{frame}
	\titlepage
\end{frame}

\begin{frame}[fragile]
	
Se tiver pressa...

\begin{bashcode}
	$ git clone https://github.com/rg3915/django1.8.git
	$ virtualenv -p python3 django1.8
	$ cd django1.8
	$ source bin/activate
	$ make initial
	$ ./manage.py runserver
\end{bashcode}

... sen\~ao, leia o tutorial.

\end{frame}

\begin{frame}\frametitle{Ementa}

\begin{itemize}
	\item MTV e ORM
	\item 1 min de Python
	\item Instala\c c\~ao
	\item Criar o ambiente
	\item Criar o projeto e a App
	\item Deploy no Heroku
\end{itemize}

\end{frame}

\begin{frame}\frametitle{Objetivo}


\begin{itemize}
	\item Criar uma lista de filmes
	\item Retornar o filme de maior bilheteria
	\item Criar um formul\'ario
	\item Ver os detalhes de cada filme
\end{itemize}

\end{frame}


\begin{frame}\frametitle{O que \'e Django?}

Segundo Django Brasil,

\

\begin{exampleblock}{}
	{\it Django \'e um framework web de alto n\'ivel escrito em Python que estimula o desenvolvimento r\'apido e limpo.}
\end{exampleblock}

\end{frame}

\begin{frame}\frametitle{O que \'e Django?}

\begin{itemize}
	\item adota o padr\~ao MTV
	\item possui ORM
	\item admin
	\item heran\c ca de templates e modelos
	\item open source
\end{itemize}

\end{frame}

\begin{frame}\frametitle{Quem usa Django?}

	\begin{figure}[h]
	  \centering
  		\includegraphics[height=.7\paperheight]{img/eu_uso_django}
	\end{figure}

\begin{center}
\url{www.djangosites.org}
\end{center}


\end{frame}


\begin{frame}\frametitle{Sites}

\begin{enumerate}
	\item \url{www.djangoproject.com/}

	\item \url{www.djangobrasil.org/} (desatualizado)

	\item \url{www.djangopackages.com/}

	\item \url{www.groups.google.com/forum/django-brasil}

	\item \url{www.pythonclub.com.br/}

	\item \url{www.github.com/rg3915/django-basic-apps}

	\item \url{www.realpython.com/blog/categories/django/}

	\item \url{www.marinamele.com/taskbuster-django-tutorial}
\end{enumerate}

\end{frame}



\begin{frame}\frametitle{MVC x MTV}

\begin{itemize}
	\item {\bf Model} - \'e o modelo, a camada de abstra\c c\~ao do banco de dados, onde acontece o ORM
	\item {\bf View} - \'e o controlador, onde acontece as regras de neg\'ocio e a comunica\c c\~ao entre a base de dados e o navegador
	\item {\bf Templates} - \'e a camada de apresenta\c c\~ao, s\~ao as p\'aginas html
\end{itemize}

\end{frame}

\begin{frame}\frametitle{MVC x MTV}
	\begin{figure}[h]
	  \centering
  		\includegraphics[height=.9\paperheight]{img/mtv1.png}
	\end{figure}
\end{frame}

\begin{frame}\frametitle{MVC x MTV}
	\begin{figure}[h]
	  \centering
  		\includegraphics[height=.9\paperheight]{img/mtv2.png}
	\end{figure}
\end{frame}


\begin{frame}\frametitle{ORM}

\begin{itemize}
	\item Modelo (Classe) = Tabela
	\item Atributos = Colunas
	\item Objetos = Tuplas (Registros)
\end{itemize}

\end{frame}

\begin{frame}[fragile]\frametitle{1 min de Python}

\begin{minted}{python}
    public static void main (String[] args){
        System.out.println("Desculpa");
    } # oops

    print("Python") # simples assim

    def soma(a, b):
        return a + b

    soma(25,9)

    lista = ['a', 10, 5.5]
    for i in lista:
        print(i)

    for i in range(10):
        print(i)

\end{minted}

\end{frame}

% ## O que você precisa?
% 
% * Python (2 ou 3)
% * Pip
% * VirtualEnv
% 
% ### Instalando Python no Windows
% 
% * Download do python: https://www.python.org/downloads/windows/
% * Configurar as vari\'aveis de ambiente (PATH)
% 
% **Leia**: "Instalando e Configurando o Python e Django no Windows" - Thiago Corôa http://pythonclub.com.br/% instalacao-python-django-windows.html
% 
% ### Pip
% 
% * Gerenciador de pacotes do python
% * https://pip.pypa.io/en/latest/installing.html#install-pip
% 
% 
% ### Instalando no Linux
% 
% 
% * Instale primeiro o **pip** http://pip.readthedocs.org/en/latest/
% 
% Primeira op\c c\~ao
% 
    % $ wget https://bootstrap.pypa.io/get-pip.py
    % $ sudo python get-pip.py
% 
% Segunda op\c c\~ao
% 
    % $ sudo apt-get install -y python-pip
% 
% * **VirtualEnv** https://virtualenv.pypa.io/en/latest/
% 
% Digite
% 
    % $ sudo pip install virtualenv
    % $ # ou
    % $ sudo apt-get install -y virtualenv
% 
% 
% 
% ## O que vamos considerar no nosso projeto?
% 
% * Ambiente: venv
% * Projeto: myproject
% * App: core
% 
% 
% 
% ## Criando o ambiente
% 
% Vamos criar um ambiente usando o **Python 3**, ent\~ao digite
% 
	% $ virtualenv -p /usr/bin/python3 venv
% 
% onde ``venv`` é o nome do ambiente.
% 
% Entre na pasta
% 
	% $ cd venv
% 
% e ative o ambiente
% 
	% $ source bin/activate
% 
% Obs: todos os pacotes instalados com o ambiente ativado ser\~ao instalados dentro do ambiente e visíveis % somente nele.
% 
% **Dica**: No Linux, edite o arquivo `~/.bashrc`
% 
	% alias sa='source bin/activate;'
% 
% Assim você cria atalhos para ativar seus ambientes:
% 
	% $ sa
% 
% **Dica**: Para diminuir o caminho do prompt digite
% 
	% $ PS1="(`basename \"$VIRTUAL_ENV\"`):/\W$ "
% 
% O caminho vai ficar assim
% 
	% (venv):/venv$
% 
% Onde `(venv)` é o nome do ambiente e `:/venv$` é a pasta atual.
% 
% Para desativar o ambiente digitamos
% 
	% (venv):/venv$ deactivate
% 
% 
% 
% 
% ## Instalando Django 1.8 + django-bootstrap-form
% 
	% $ pip install django==1.8.4 django-bootstrap-form
% 
% https://github.com/tzangms/django-bootstrap-form
% 
% Vendo o que foi instalado
% 
	% $ pip freeze
	% Django==1.8.4
	% django-bootstrap-form==3.2
% 
% Crie o *requirements.txt* (os ingredientes do bolo)
% 
	% $ pip freeze > requirements.txt
% 
% 
% 
% ## Criando o projeto e a App
% 
% https://docs.djangoproject.com/en/1.8/intro/tutorial01/
% 
% Para criar o **projeto** digite
% 
	% $ django-admin.py startproject myproject .
% 
% repare no ponto final do comando, isto permite que o arquivo `manage.py` fique na pasta "principal", pasta % **venv**.
% 
% Criando a **app**
% 
	% $ python manage.py startapp core
	% ou
	% $ ./manage.py startapp core
	% ou
	% $ manage startapp core
% 
% **Dica**: para funcionar o último comando você deve editar o `~/.bashrc`
% 
	% $ alias manage='python $VIRTUAL_ENV/manage.py'
% 
% O que temos até aqui?
% 
	% $ tree myproject; tree core
% 
% 
	% .
	% ├── manage.py
	% ├── myproject
	% │   ├── __init__.py
	% │   ├── settings.py
	% │   ├── urls.py
    % │   └── wsgi.py
	% ├── core
	% │   ├── admin.py
	% │   ├── __init__.py
	% │   ├── models.py
	% │   ├── tests.py
    % │   └── views.py
% 
% 
% ## Django funcionando em nível 0
% 
% Criando a primeira migra\c c\~ao
% 
	% $ python manage.py migrate
% 
% Obs: o comando ``migrate`` se chamava ``syncdb`` e s\'o era capaz de criar novas tabelas no banco de dados. % J\'a o ``migrate`` consegue remover e alterar tabelas. Criado baseado nas funcionalidades do Django South.
% 
	% $ python manage.py runserver
% 	
% Por padr\~ao ele est\'a rodando na porta 8000
% 
% http://localhost:8000/ ou http://127.0.0.1:8000/
% 
% ou
% 
	% $ python manage.py runserver <PORTA>
	% $ python manage.py runserver 8080
% 
% http://localhost:8080/
% 
% ![image](img/it_worked.png)
% 
% ### Django South
% 
% * Detecta alterações nos models.py e gera scripts para ajustar o banco de dados
% * Faz "versionamento" de bases de dados
% * Permite migra\c c\~ao de bases dados de um SGBD para outro
% 
% ## O mínimo - nível 1: settings, views, urls
% 
% 
	% .
	% ├── myproject
	% │   ├── ...
	% │   ├── settings.py
	% │   └── urls.py
	% ├── core
	% │   ├── ...
    % │   └── views.py
% 
% 
% ### Editando settings.py
% 
	% INSTALLED_APPS = (
	    % ...
	    % 'core',
	% )
% 
% 
% ### Editando views.py
% 
	% # -*- coding: utf-8 -*-
	% # from django.shortcuts import render
	% from django.http import HttpResponse
% 
% 
	% def home(request):
	    % return HttpResponse('<h1>Django</h1><h3>Bem vindo ao Grupy-SP</h3>')
% 
% 
% 
% 
% ### Editando urls.py
% 
	% from django.conf.urls import include, url
% 
	% urlpatterns = [
	    % url(r'^$', 'core.views.home'),
	    % url(r'^admin/', include(admin.site.urls)),
	% ]
% 
% Ou
% 
	% from django.conf.urls import patterns, include, url
% 
	% urlpatterns = patterns(
		% 'core.views',
	    % url(r'^$', 'home'),
	    % url(r'^admin/', include(admin.site.urls)),
	% )
% 
% ![image](img/HttpResponse.png)
% 
% 
% 
% ### Admin
% 
	% $ python manage.py createsuperuser --username='admin' --email=''
% 
% ![image](img/admin1.png)
% 
% ![image](img/admin2.png)
% 
% 
% 
% 
% ## Tocando o barco
% 
% 
% ## Editando settings.py
% 
% 
	% LANGUAGE_CODE = 'pt-br'
% 
	% TIME_ZONE = 'America/Sao_Paulo'
% 
	% LOGIN_URL = '/admin/login'
% 
% 
% ## Testes
% 
% ![image](img/teste03.png)
% 
% ### Teste: Verificar se existe a p\'agina *index.html*.
% 
	% from django.test import TestCase
% 
% 
	% class HomeTest(TestCase):
% 
	    % def setUp(self):
	        % self.resp = self.client.get('/')
% 
	    % def test_get(self):
	        % ''' get / deve retornar status code 200. '''
	        % self.assertEqual(200, self.resp.status_code)
% 
	    % def test_template(self):
	        % ''' Home deve usar template index.html '''
	        % self.assertTemplateUsed(self.resp, 'index.html')
% 
% 
% **Leia**: "pytest: escreva menos, teste mais" - Erick Wilder de Oliveira - https://goo.gl/8E9FB1 
% 
% 
% ## Editando views.py
% 
	% from django.shortcuts import render
	% # from django.http import HttpResponse
% 
	% # def home(request):
	% #     return HttpResponse('<h1>Django</h1><h3>Bem vindo ao Grupy-SP</h3>')
% 
	% def home(request):
	    % return render(request, 'index.html')
% 
% 
% ## Criando o index.html
% 
% Estando na pasta `venv` digite
% 
	% $ mkdir -p core/templates
	% $ echo "<html><body><h1>Tutorial Django</h1><h3>Bem vindo ao Grupy-SP</h3></body></html>" > core/% templates/index.html
% 
% 
% ## Editando models.py
% 
% **B\'asico**: Filmes
% 
% ![image](img/diagram.png)
% 
% 
	% # -*- coding: utf-8 -*-
	% from django.db import models
% 
% 
	% class Distributor(models.Model):
	    % distributor = models.CharField('distribuidor', max_length=50, unique=True)
% 
	    % class Meta:
	        % ordering = ['distributor']
	        % verbose_name = 'distribuidor'
	        % verbose_name_plural = 'distribuidores'
% 
	    % def __str__(self):
	        % return self.distributor
% 
% 
	% class Category(models.Model):
	    % category = models.CharField('categoria', max_length=50, unique=True)
% 
	    % class Meta:
	        % ordering = ['category']
	        % verbose_name = 'categoria'
	        % verbose_name_plural = 'categorias'
% 
	    % def __str__(self):
	        % return self.category
% 
% 
	% class Movie(models.Model):
	    % movie = models.CharField('filme', max_length=100)
	    % category = models.ForeignKey(
	        % 'Category', verbose_name='categoria', related_name='movie_category')
	    % distributor = models.ForeignKey(
	        % 'Distributor', verbose_name='distribuidor', related_name='movie_distributor')
	    % raised = models.DecimalField('arrecadou', max_digits=4, decimal_places=3)
	    % liked = models.BooleanField('gostou', default=True)
	    % release = models.DateTimeField(u'lançamento')
% 
	    % class Meta:
	        % ordering = ['-release']
	        % verbose_name = 'filme'
	        % verbose_name_plural = 'filmes'
% 
	    % def __str__(self):
	        % return self.movie
% 
% 
% ## Tipos de campos
% 
% * BooleanField
% * CharField
% * DateField
% * DateTimeField
% * DecimalField
% * DurationField
% * EmailField
% * FileField
% * FloatField
% * ImageField
% * IntegerField
% * NullBooleanField
% * PositiveIntegerField
% * PositiveSmallIntegerField
% * SlugField
% * SmallIntegerField
% * TextField
% * TimeField
% * ForeignKeyField
% * ManyToManyField
% * OneToOneField
% 
% 
% https://docs.djangoproject.com/en/1.8/ref/models/fields/
% 
% 
% 
% ## Atualizando o banco
% 
	% $ python manage.py makemigrations
	% $ python manage.py migrate
% 
% 
% ## shell
% 
% Explorando um pouco as queryset.
% 
	% $ python manage.py shell
	% Python 3.4.0 (default, Jun 19 2015, 14:18:46) 
	% [GCC 4.8.2] on linux
	% Type "help", "copyright", "credits" or "license" for more information.
	% (InteractiveConsole)
	% >>> 
% 
% Precisamos importar o models.
% 
	% >>> from core.models import Distributor, Category, Movie
% 
% Todos os comandos est\~ao em shell/shell.py
% 	
	% $ manage shell < shell/distributors.py
	% $ manage shell < shell/movies.py
% 
% 
% https://docs.djangoproject.com/en/1.8/ref/models/querysets/
% 
% https://pt.wikipedia.org/wiki/Lista_de_filmes_de_maior_bilheteria
% 
% ## Admin
% 
	% from django.contrib import admin
	% from .models import Distributor, Category, Movie
% 
	% admin.site.register(Distributor)
	% admin.site.register(Category)
	% admin.site.register(Movie)
% 
% 
% ## Criando os templates
% 
	% $ mkdir core/templates/core
	% $ touch core/templates/{base.html,menu.html}
	% $ touch core/templates/core/{movie_list.html,movie_detail.html,movie_form.html}
% 
% Temos
% 
	% core
	% ├── admin.py
	% ├── models.py
	% ├── templates
	% │   ├── base.html
	% │   ├── index.html
	% │   ├── menu.html
	% │   └── core
	% │       ├── movie_detail.html
	% │       ├── movie_form.html
	% │       └── movie_list.html
	% ├── tests.py
	% └── views.py
% 
% ### Vari\'aveis
% 
% Acessando objetos
% 
	% {{ objeto }}
% 
% Acessando atributos
% 
	% {{ objeto.atributo }}
% 
% Tags
% 
	% 
% 
% Exemplo:
% 
	% 
% 
	% 
% 
	% 
		% {{ item.atributo }}
	% 
% 
% Vamos editar:
% 
% **menu.html**
% 
	% <a href="">Home</a>
% 
% **base.html**
% 
	% 
% 
	% 
		% html
	% 
% 
% ## Herança de Templates
% 
% 
% **index.html**
% 
	% 
% 
	% 
		% <div class="container">
			% <div class="jumbotron">
				% <h1>Tutorial Django</h1>
				% <h3>Bem vindo ao Grupy-SP</h3>
			% </div>
		% </div>
	% 
% 
% 
% 
% 
% 
% **movie_list.html**
% 
	% 
		% <ul>
			% <li>{{ item.movie }}</li>
			% <li>{{ item.category }}</li>
			% ...
		% </ul>
	% 
% 
% 
% **movie_list.html (completo)**
% 
	% 
% 
	% 
% 
		% 
			% <table>
			    % <tbody>
			    % 
			        % <tr>
		            	% <td>{{ movie.movie }}</td>
		            	% <td>{{ movie.category }}</td>
		            	% <td>{{ movie.distributor }}</td>
		            	% <td>U$ {{ movie.raised }}</td>
		            	% <td>{{ movie.release|date:"d/m/Y" }}</td>
		            	% 
							% <td><span class="glyphicon glyphicon-ok-sign" style="color: #44AD41"></span></% td>
						% 
							% <td><span class="glyphicon glyphicon-minus-sign" style="color: #DE2121"></% span></td>
						% 
			        % </tr>
			    % 
			    % </tbody>
			% </table>
		% 
			% <p class="alert alert-warning">Sem itens na lista.</p>
		% 
	% </div>
	% 
% 
% 
% ![image](img/lista.png)
% 
% 
% 
% **movie_detail.html**
% 
	% {{ object.movie }}
% 
% http://getbootstrap.com/
% 
% http://getbootstrap.com/examples/theme/
% 
% http://www.layoutit.com/
% 
% 
% 
% ## Visualizando os dados com json
% 
% views.py
% 
	% def movie_list_json(request):
	    % movies = Movie.objects.all()
	    % s = serializers.serialize("json", movies)
	    % return HttpResponse(s)
% 
% urls.py
% 
    % url(r'^movie/json$', 'movie_list_json', name='movie_list_json'),
% 
% 
% 
% ## Editando a views.py
% 
	% def movie_list(request):
	    % movies = Movie.objects.all()
	    % context = {'movies': movies}
	    % return render(request, 'core/movie_list.html', context)
% 
% 
% 
% ### Class Based View
% 
% https://docs.djangoproject.com/en/1.8/topics/class-based-views/
% 
% https://ccbv.co.uk/
% 
% **Leia**: "Django Class Based Views - o que s\~ao e por que usar" - Caio Carrara https://goo.gl/xnfqx1
% 
% https://speakerdeck.com/cacarrara/django-class-based-views
% 
% 
% ### Editando o views.py para lista
% 
	% class MovieList(ListView):
	    % template_name = 'core/movie_list.html'
	    % model = Movie
	    % context_object_name = 'movies'
% 
% 
% 
% 
% ## Formul\'arios
% 
% ### Editando o views.py para formul\'ario
% 
% 
	% class MovieCreate(CreateView):
	    % template_name = 'core/movie_form.html'
	    % model = Movie
	    % fields = '__all__'
	    % success_url = reverse_lazy('movie_list')
% 
% 
% 
% ### Editando *movie_form.html*
% 
% ![image](img/form.png)
% 
% Existem v\'arias formas de se criar um formul\'ario, qual deles eu uso?
% 
% 1. Fazendo tudo na m\~ao com html puro
% 
% ```
	% 
% 
	% 
% 
	% <div class="container">
		% <form class="form-horizontal" action="." method="POST">
		    % <legend>Cadastrar</legend>
		    % 
% 
		    % <div class="form-group">
		    	% <label for="id_movie">Filme</label>
		    	% <input type="text" id="id_movie" name="movie" class="form-control">
		    % </div>
% 			
		    % <div class="form-group">
		    	% <label for="id_category">Categoria</label>
% 		    	<input type="text" id="id_category" name="category" class="form-control" placeholder="Tem % que usar select">
		    % </div>
% 
			% <!-- ... -->
% 
			% <div class="form-group">
		      % <div class="col-sm-10 col-sm-offset-2">
		        % <button type="submit" id="id_submit" class="btn btn-primary">Salvar</button>
		      % </div>
		    % </div>
		% </form>
	% </div>
% 
	% 
% ```
% 
% 
% 2. Usando as tags do Django
% 
% ```
	% {{ form }}
% 
	% {{ form.as_p }}
% 
	% {{ form.as_ul }}
% 
	% {{ form.as_table }}
% ```
% 
% Nosso formul\'ario
% 
	% 
% 
	% 
		% <form action="" method="POST">
			% 
			% {{ form.as_p }}
		% </form>
	% 
% 
% 
% 
% 3. Usando {{ field.label }} e {{ field }}
% 
% ```
	% 
	  % <div class="form-group">
	    % <div class="control-label col-sm-2">
	      % {{ field.errors }}
	      % {{ field.label }}
	    % </div>
	    % <div class="col-sm-2">
	      % {{ field }}
	    % </div>
	  % </div>
    % 
% ```
% 
% 
% 4. Usando bibliotecas como o django-bootstrap-form
% 
% ```
	% 
% 
	% 
% 
	% 
% 
	% <div class="container">
		% <form class="form-horizontal" action="." method="POST">
		    % <legend>Cadastrar</legend>
		    % 
		    % {{ form.movie|bootstrap_horizontal }}
		    % {{ form.category|bootstrap_horizontal }}
		    % {{ form.distributor|bootstrap_horizontal }}
		    % {{ form.raised|bootstrap_horizontal }}
		    % {{ form.liked|bootstrap_horizontal }}
		    % {{ form.release|bootstrap_horizontal }}
% 
			% <div class="form-group">
		      % <div class="col-sm-10 col-sm-offset-2">
		        % <button type="submit" id="id_submit" class="btn btn-primary">Salvar</button>
		      % </div>
		    % </div>
		% </form>
	% </div>
% 
	% 
% ```
% 
% ![image](img/form4.png)
% 
% ### Editando o urls.py
% 
    % url(r'^movie/add/$', MovieCreate.as_view(), name='movie_add'),
% 
% 
% ## Um pouco de Selenium
% 
	% $ python selenium/selenium_movie.py	
% 
% **Leia**: "Testes com Selenium" - Jayme Neto https://goo.gl/sO7gLB
% 
% ## Carregando dados de um json
% 
	% $ python manage.py loaddata fixtures.json
% 
% 
% ## Visualizando os Detalhes
% 
% views.py
% 
	% class MovieDetail(DetailView):
	    % template_name = 'core/movie_detail.html'
	    % model = Movie
% 
% urls.py
% 
    % url(r'^movie/(?P<pk>\d+)/$', MovieDetail.as_view(), name='movie_detail'),
% 
% movie_detail.html
% 
% ```
	% 
% 
	% 
		% <div class="container">
			% <div class="row">
% 
				% <div class="col-sm-6 col-md-4">
					% <div class="list-group">
						% <h1>{{ object.movie}}</h1>
						% <div class="list-group-item">
							% <h4>{{ object.category }}</h4>
						% </div>
						% <div class="list-group-item">
							% <h4>{{ object.distributor }}</h4>
						% </div>
						% <div class="list-group-item">
							% <h3>U$ {{ object.raised }}</h3>
						% </div>
						% <div class="list-group-item">
							% <h4>{{ object.release|date:"d/m/Y" }}</h4>
						% </div>
					% </div>
				% </div>
			% </div>
		% </div>
	% 
% ```
% 
% models.py
% 
	% class Movie(models.Model):
		% ...
		% def get_absolute_url(self):
	        % return reverse_lazy('movie_detail', kwargs={'pk': self.pk})
% 
% 
% 
% movie_list.html
% 
% ```
	% <td><a href="{{ movie.get_absolute_url }}">{{ movie.movie }}</a></td>
% ```
% 
% ![image](img/detalhes.png)
% 
% 
% ## Resumo dos comandos
% 
	% $ django-admin.py startproject myproject .
	% $ python manage.py startapp core
	% $ python manage.py migrate
	% $ python manage.py makemigrations
	% $ python manage.py migrate
	% $ python manage.py createsuperuser --username='admin' --email=''
	% $ python manage.py test
	% $ python manage.py shell
	% $ python manage.py runserver
	% $ python manage.py dumpdata core --format=json --indent=2 > fixtures.json
	% $ python manage.py loaddata fixtures.json
% 
% 
% 
% 
% ## Deploy no Heroku
% 
% Você deve ter uma conta no **GitHub** e no **Heroku**.
% 
% ### Instale o heroku toolbelt
% 
	% $ wget -O- https://toolbelt.heroku.com/install-ubuntu.sh | sh
% 
% https://toolbelt.heroku.com/debian
% 
% ### Crie o Runtime e o Procfile
% 
	% $ heroku login
	% $ echo "python-3.4.0" > runtime.txt
	% $ heroku create django18grupy
	% $ echo "web: gunicorn myproject.wsgi" > Procfile
	% $ pip install dj-static gunicorn psycopg2
	% $ pip freeze > requirements.txt
% 
% ### Edite o wsgi.py
% 
	% import os
	% os.environ.setdefault("DJANGO_SETTINGS_MODULE", "myproject.settings")
% 
	% from django.core.wsgi import get_wsgi_application
	% from dj_static import Cling
% 
	% application = Cling(get_wsgi_application())
% 
% ### Edite o settings.py
% 
	% DATABASES = {
	    % 'default': dj_database_url.config(
	        % default='sqlite:///' + os.path.join(BASE_DIR, 'db.sqlite3'))
	% }
% 
% Faça o push no GitHub.
% 
	% $ git add .
	% $ git commit -m "config to heroku"
	% $ git push origin master
% 
% Agora, os comandos do heroku
% 
	% $ git push heroku master --force
	% $ heroku ps:scale web=1
	% $ heroku labs:enable user-env-compile
	% $ heroku pg
	% $ heroku run python manage.py makemigrations
	% $ heroku run python manage.py migrate
	% $ heroku pg
	% $ heroku run python manage.py createsuperuser --username='admin' --email=''
	% $ heroku run python manage.py loaddata fixtures.json
	% $ heroku open
% 
% https://devcenter.heroku.com/articles/getting-started-with-django
% 
% ## Awesome Django
% 
% https://github.com/rosarior/awesome-django
% 
% ### Admin interface
% 
% **django-grappelli**
% 
% https://github.com/sehmaschine/django-grappelli/
% 
% **django-admin-bootstrap**
% 
% https://github.com/django-admin-bootstrap/django-admin-bootstrap
% 
% **django-material**
% 
% https://github.com/viewflow/django-material
% 
% ### Database
% 
% **dj-database-url**
% 
% https://github.com/kennethreitz/dj-database-url/
% 
% ### Debugging
% 
% **django-debug-toolbar**
% 
% https://github.com/django-debug-toolbar/django-debug-toolbar/
% 
% ### Forms
% 
% **django-bootstrap-form**
% 
% https://github.com/tzangms/django-bootstrap-form/
% 
% **django-bootstrap3**
% 
% https://github.com/dyve/django-bootstrap3/
% 
% **django-crispy-forms**
% 
% https://github.com/maraujop/django-crispy-forms/
% 
% **django-floppyforms**
% 
% https://github.com/gregmuellegger/django-floppyforms/
% 
% **django-autocomplete-light**
% 
% https://github.com/yourlabs/django-autocomplete-light/
% 
% ### RESTful API
% 
% **django-rest-framework**
% 
% http://www.django-rest-framework.org/
% 
% ### Migrations
% 
% **South**
% 
% https://bitbucket.org/andrewgodwin/south/src/
% 
% ### Model Extensions
% 
% **django-aggregate-if**
% 
% https://github.com/henriquebastos/django-aggregate-if/
% 
% ### Testing
% 
% **model-mommy**
% 
% https://github.com/vandersonmota/model_mommy/
% 
% **mixer**
% 
% https://github.com/klen/mixer
% 
% ### Other
% 
% **django-extensions**
% 
% https://github.com/django-extensions/django-extensions/
% 
% 
% ## Livros
% 
% * Django Essencial de Julia Elman da Novatec
% 
% http://www.novatec.com.br/livros/django/
% 
% * Two Scoops of Django 1.8 de Daniel and Audrey Roy Greenfeld (Py Danny)
% 
% http://twoscoopspress.org/pages/current-django-books
% 
% http://djangoteca.info/livros/django/
% 
% * Django Book online
% 
% http://www.djangobook.com/en/2.0/index.html
% 
% 
% 
% 
% ## Cursos
% 
% * Django presencial na CTNovatec (S\~ao Paulo) com Júlio C. Melanda, dias 03 e 04/10/15 (S\'ab e Dom)
% 
% http://ctnovatec.com.br/cursos/trilha-python/curso-de-django/
% 
% * Welcome to the Django (online) com Henrique Bastos, em 2015
% 
% http://welcometothedjango.com.br/
% 
% * PyCursos (online) Jornada Django com Gileno Filho
% 
% http://pycursos.com/django/
% 
% 
% 
% ## YouTube
% 
% * Python para Zumbis - https://goo.gl/swsHmw
% * Django para Iniciantes por Allisson Azevedo - https://goo.gl/38ttOb
% * CodingEntrepreneurs Try Django 1.8 - https://goo.gl/HNxRou



\end{document}